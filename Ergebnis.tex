\section{Ergebnis und Ausblick }

Im Rahmen dieses Independent Courseworks wurde versucht eine Umsetzung der Ideen des Masterprojektes neeedo.com f"ur iOS 9 unter Verwendung von Swift2 zu erreichen. 
Dies sollte auf Grundlage der existierenden Android Applikation geschehen, die zuvor nach Kriterien der Usability analysiert und Umgestaltet wurde. 

Bis zum Ende des Projektzeitraumes war es m"oglich eine prototypische Umsetzung der iOS-Applikation zu erreichen, in der ein Gro"steil der angestrebten Funktionalit"aten implementiert werden konnte.

Das Projekt kann in seinem aktuellen Stand unter \url{https://github.com/Cneubauern/neeedoIOS} heruntergeladen werden.

\subsection{Ergebnis}

Die folgenden Funktionen konnten zum jetzigen Zeitpunkt umgesetzt werden.
\subsubsection*{Benutzer}
\begin{description}
\item [+]Benutzer anlegen \vspace{-0,2cm}
\item [+]Benutzer l"oschen \vspace{-0,2cm}
\item [+]Benutzer ein/ausloggen \vspace{-0,2cm}
\end{description}

\subsubsection*{Angebote}
\begin{description}
\item [+]Angebote erstellen \vspace{-0,2cm}
\item [+]Angebote aktualisieren  \vspace{-0,2cm}
\item [+]Angebote l"oschen \vspace{-0,2cm}
\item [+]Eigene Angebote anzeigen \vspace{-0,2cm}
\end{description}

\subsubsection*{Gesuche}
\begin{description}
\item [+]Gesuche erstellen \vspace{-0,2cm}
\item [+]Gesuche aktualisieren \vspace{-0,2cm}
\item [+]Gesuche l"oschen \vspace{-0,2cm}
\item [+]Eigene Gesuche anzeigen \vspace{-0,2cm}
\end{description}

\subsubsection*{Favoriten}
\begin{description}
\item [+]Favoriten hinzuf"ugen \vspace{-0,2cm}
\item [+]Favoriten entfernen \vspace{-0,2cm}
\item [+]Favoriten anzeigen \vspace{-0,2cm}
\end{description}

\subsubsection*{Favoriten}
\begin{description}
\item [+]Favoriten hinzuf"ugen \vspace{-0,2cm}
\item [+]Favoriten entfernen \vspace{-0,2cm}
\item [+]Favoriten anzeigen \vspace{-0,2cm}
\end{description}

Aktuell gibt es noch Probleme beim Matching, dieses f"uhrt unter bestimmten Umst"anden zu Abst"urzen, sowie beim ImageUpload von der Camera. 
Das Messaging System wurde als Ansicht implementiert ist aber noch nicht funktionsf"ahig 
Das von der neeedo-API angebotene Tag-Suggestion-System wurde aus zeitgr"unden nicht mehr bearbeitet. 

\subsection{Ausblick}

F"ur die Zukunft gilt es vor allem an der Konsistenz und am Design zu arbeiten. Es sind noch viele kleine und Mittelgro"se Bugs zu beheben und die beschriebenen Funktionalit"aten vollst"andig umzusetzen. 

\subsection{Res"umee}

F"ur mich war dieses Projekt eine interessante Herausforderungen. Zwar habe ich in der Vergangenheit bereits Erfahrungen mit der iOS Entwicklung gemacht, doch war Swift2 noch sehr neu f"ur mich. 
Ebenso war dies die erste iOS-App mit Netzwerkanbindung die ich selbst entwickelt habe. 
Ich konnte im Verlauf des Projektes viel lernen und neue Erfahrungen machen. 
