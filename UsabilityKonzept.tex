\section{Usability-Konzept}

Basieren auf den Ergebnissen der IST-Analyse kann ein Konzept erarbeitet werden, dass die erkannten St"arken aufgreift und versucht die erkannten Schw"achen zu verbessern.

\subsection{Vorbetrachtung}

Aus der IST-Analyse und der gegebenen API lassen sich die folgenden Funktionen, bzw. Use Cases ableiten, die in der App umgesetzt werden sollen.

\subsubsection{UseCases}

\subsubsection*{Benutzer}
\begin{itemize}
\item Benutzer anlegen \vspace{-0,2cm}
\item Benutzer l"oschen \vspace{-0,2cm}
\item Benutzer ausloggen \vspace{-0,2cm}
\end{itemize}

\subsubsection*{Angebote}
\begin{itemize}
\item Angebote erstellen \vspace{-0,2cm}
\item Angebote aktualisieren  \vspace{-0,2cm}
\item Angebote l"oschen \vspace{-0,2cm}
\item Eigene Angebote anzeigen \vspace{-0,2cm}
\end{itemize}

\subsubsection*{Gesuche}
\begin{itemize}
\item Gesuche erstellen \vspace{-0,2cm}
\item Gesuche aktualisieren \vspace{-0,2cm}
\item Gesuche l"oschen \vspace{-0,2cm}
\item Eigene Gesuche anzeigen \vspace{-0,2cm}
\end{itemize}

\subsubsection*{Favoriten}
\begin{itemize}
\item Favoriten hinzuf"ugen \vspace{-0,2cm}
\item Favoriten entfernen \vspace{-0,2cm}
\item Favoriten anzeigen \vspace{-0,2cm}
\end{itemize}

\subsubsection*{Matching}
\begin{itemize}
\item Zeige passende Angebote \vspace{-0,2cm}
\end{itemize}

\subsubsection{Wie sollen diese umgesetzt werden.}

\subsubsection*{Benutzer}

Wie in der IST- Analyse bereits erw"ahnt, ist der UseCase des Nutzer-Anlegens, bzw. des User Logins bereits in einer praktikablen Weise umgesetzt.
Diese soll auch in der hier entstehenden App "ubernommen werden. 

Das Ausloggen ist in der Android Applikation durch einen Button in der Men"uleiste oben umgesetzt, jedoch besteht keine M"oglichkeit seinen Account zu l"oschen. 

In der geplanten App sollen diese beiden Funktionen zusammen in einem Men"u angeboten werden. 

\subsubsection*{Angebote, Gesuche erstellen/bearbeiten/l"oschen}

Grunds"atzlich l"asst sich an der Umsetzung des Erstellens und Aktualisierens der Angebote und Gesuche nicht viel aussetzen, au"ser einzelner Designfehler. 
Die Umsetzung mit Formularen l"asst sich nur schwer umgehen und w"urde kaum Vorteile bieten. 

Allerdings sind diese Formulare im aktuellen Zustand nicht besonders gut strukturiert und m"ussen angepasst werden, um Uneindeutigkeiten zu umgehen. 

Das L"oschen der Elemente geschieht durch einen Button auf jeweiligen Ansicht.
Dies l"asst sich ebenfalls aufgreifen. 

In iOS haben sich inzwischen aber auch, gerade in Tabellendarstellungen, Alternative M"oglichkeiten entwickelt. 
Diese sollen hier ebenfalls zum Einsatz kommen. 

\subsubsection*{Favoriten}

Das Hinzuf"ugen von Favoriten kann nur aus der jeweiligen Offer Darstellung geschehen.  Daran l"asst sich nichts "andern, da die API aktuell nur Anfgebote als Favorit zul"asst. 
In der aktuellen Umsetzung geschieht das durch einen \enquote{Stern}-Button auf der jeweiligen Karte. 

Der Stern ist eine bekannte Art der Darstellung f"ur Markierungen und Favoriten. 
Dies kann so "ubernommen werden. 

\subsubsection*{Favoriten, (Eigene) Angebote, (Eigene) Gesuche anzeigen}

Um eine "Ubersicht seiner eigenen Favoriten, Angebote und Gesuche darzustellen wurde in der Android Applikation eine Listen-Darstellung gew"ahlt. 
Diese zeigt sich als funktional und sinnvoll. 

Die Detail-Darstellung der einzelnen Elemente geschieht durch eine Karten-Darstellung auf der alle relevanten Informationen angezeigt werden.
Die grunds"atzliche Art der Darstellung dieser Elemente ist passend.
In dieser Arbeit sollen lediglich "Anderungen an inneren Struktur der Elemente durchgef"uhrt werden.

Im Rahmen der "Anderungen an der M"en"f"uhrung wird sich der Zugang zu diesen Ansichten "andern.

\subsubsection*{Matching}

In der Android Applikation werden die gematchten Angebote in der Tinder-"ahnlichen Stapeldarstellung dargestellt. 
Diese Darstellung entwickelt aktuell eine gewisse Beliebtheit und wird auch in vielen anderen Apps angeboten.
Es l"asst sich auch kein grundlegender Fehler hieran erkennen daher kann sie in einer angepassten Weise "ubernommen werden.

\subsubsection*{Men"u}

Eine gro"se Schw"ache der Android Applikation war das viel zu gro"se Seitenmen"u, dass keinen besonderen Zweck erf"ullt. 
Da die angebotenen Funktionen hier sehr begrenzt sin bietet es sich an dieses durch eine TabBar zu ersetzen. 
Diese hat den Vorteil, dass die Funktionen immer pr"asent sind und keinen zus"atzlichen Platz ben"otigen
