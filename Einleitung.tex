\section{Einleitung}
\subsection{Ausgangslage}

Im Wintersemester 2014/15 und Sommersemester '15 entstand in Zusammenarbeit mit der Firma Commercetools das Masterprojekt \enquote{neeedo.com}. 
Die Idee dieses Projektes wurde als next-generation e-Commerce tituliert und sollte, alternative Ideen und moderne Technologien in das e-Commerce Umfeld bringen. 

\subsubsection{Die Firma Commercetools}  

Die Firma Commercetools entwickelt und vertreibt die e-Commerce-Plattform "Sphere.io", heute "commerceTools platform". Dies ist eine Commerce-as-a-Service Platform, die es dem Anwender erlaubt auf einfache Weise einen (Online-) Shop zu erstellen und zu verwalten. F"ur diese Plattform steht ebenfalls eine Entwickler-API zur Verf"ugung, auf der dieses Projekt aufbaut.

\subsubsection{Das Ergebnis}

Das Ergebnis des Projektes \enquote{neeedo.com} war eine Verkaufsplattform, in der die Idee der Suche/Biete Karten, bekannt vom klassischen \enquote{Schwarzen Brett}, aufgegriffen und in einem modernen Gewand pr"asentiert wird. 
Es wurden dabei neue Ans"atze f"ur das Suchen und Anbieten von Produkten umgesetzt. Ein intelligentes Matching-System hilft den Nutzern dabei das zu finden was sie suchen.

Umgesetzt wurde das Projekt mithilfe einer selbst entwickelten Scala/Play-API, die die Verbindung zum Commercetools-Service herstellt, der die Datenhaltung und weitere interne Funktionen bereitstellt. 
In dieser API finden zudem alle Operationen zum Suchen und Erstellen von Artikeln statt, sowie das Matching und andere Funktionen. 
Welche Funktionen von der API genau erf"ullt werden wird in der Umsetzung ersichtlich.

Auf dieser API aufbauend wurden eine WebApplikation und eine native Android Applikation umgesetzt, die als Schnittstelle zum Benutzer dienen. 

\subsection{Idee}

W"ahrend des Projekts stand immer wieder die Entwicklung einer zus"atzlichen iOS Applikation im Raum. 
Diese wurde aber schlussendlich nicht umgesetzt.

Dies soll nun im Rahmen dieses \enquote{Independent Courseworks} nachgeholt werden. 

Da, wie die Analyse dieser zeigen wird, markante Schw"achen in der Usability der Android Applikation existieren, soll ein Schwerpunkt dieser Arbeit sein diese zu verbessern bzw. ein neues Konzept zu entwickeln. Dies ist auch notwendig, da die aktuellen iOS-Ger"ate abweichende Verhalten im Vergleich zu Android Ger"aten aufweisen.
